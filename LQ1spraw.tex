\documentclass[11pt,a4paper]{article}
\usepackage[T1]{fontenc}
\usepackage[polish]{babel}
\usepackage[utf8]{inputenc}
\usepackage{lmodern}
\usepackage{geometry}
\usepackage{enumitem}
\selectlanguage{polish}
\usepackage{graphicx}
\DeclareGraphicsExtensions{.pdf,.png,.jpg}
\usepackage{multicol}


\newgeometry{tmargin=2cm, bmargin=2cm, lmargin=2cm, rmargin=2cm}
\title{Laboratorium SRD\\LQ1}
\date{10 maja 2014}
\author{Katarzyna Olszewska, Łukasz Korpal}
\renewcommand*\thesection{\arabic{section}}
\setcounter{secnumdepth}{5}
\setcounter{tocdepth}{5}
\setlength{\parindent}{0pt}
\begin{document}
\maketitle
\newpage

\section*{Zadanie 8-1}
Sterowany obiekt ma 2 zmienne stanu opisane następującymi równaniami

\begin{displaymath}
x[1]_{k+1} = 1.8x[1]_k + w[1]_k + 0.01 \cdot u[1]_k \\
x[2]_{k+1} = 0.2x[2]_k + 0.3x[1]_k (1 - u[1]_k ) - 5
\end{displaymath}
gdzie
\begin{displaymath}
	x[i]_k - i-ta współrzędna wektora stanu w chwili k 
\end{displaymath}
\begin{displaymath}
u[i]_k - i-ta współrzędna wektora sterowania w chwili k
\end{displaymath}
\begin{displaymath}
w[i]_k - i-ta współrzędna wektora zakłócenia w chwili k
\end{displaymath}
Zakłócenia w kolejnych chwilach są niezależne i mają rozkład stacjowany o wartosci oczekiwanej
\begin{displaymath}
E_w 0 0.1
\end{displaymath}

Celem sterowania jest osiągnięcie jak najmniejszej wartosci oczekiwanej
\begin{displaymath}
J = \sum_{k=0}^{\infty}{F_k(x_{k+1},u_k)}
\end{displaymath}

Gdzie:
\begin{displaymath}
F(x_{k+1}, u_k) = 0.1 (g(u[1]_k))^2 (x[1]_{k+1})^2 + x[2]_{k+1}^2 + 0.1u[1]_k^2
\end{displaymath}
\begin{displaymath}
g(z)=
\end{displaymath}
\begin{displaymath}
	0.4 dla z<0.4
	\end{displaymath}
	\begin{displaymath}
	z dla 0.4 \geq z \geq 1
	\end{displaymath}
	\begin{displaymath}
	1 dla 1<z
\end{displaymath}

\section{Założenia projektowe}
Początkowo wyznaczamy punkt optymalny zadania liniowego za pomocą funkcji fmincon Matlaba. Następnie dokonujemy obliczenia sterowań i stanów dla nowego modelu liniowego i uruchamiamy symulację, mającą na celu sprawdzenie poprawnosci (zbieżnosci do punktu stacjonarnego) rozwiązania zadania.

\section{Funkcje}
\subsection{Funkcja główna}
Wszystkie główne zadania realizuje funkcja lq1.m
\subsubsection{Funkcja w matlabie}
Z uwagi na długość funkcji, została ona załączona w oddzielnym pliku.

\subsubsection{Opis}
Funkcja ta zawiera dwie części - modelową i symulacyjną.\\
Początkowo zleca wyliczenie punktu optymalnego, a następnie zleca wyliczanie optymalnego sterowania.\\
W końcu uruchamia symulację i wyswietla wykres zbieznosci stanow do punktu stacjonarnego

\subsection{fun1}
Funkcja ta dla zadanych macierzy stanów i sterowań wyodrębnia aktualnie potrzebne i uruchamia obliczanie wskaźnika jakosci dla tych danych.

\subsection{Model liniowy}
Funkcja model lin służy znalezieniu początkowego punktu optymalnego - wykorzystywana w funkcji fmincon.

\subsection{Nowy model liniowy}
Funkcja model lin nowy ta służy budowie własciwego modelu, wykorzystywanego przy obliczaniu optymalnej strategii decyzyjnej.

\subsection{Sterowanie optymalne}
Funkcja ster opt ta wyznacza optymalne sterowanie dla zadanych danych.

\subsection{Stany}
Funkcja transf ta służy wyznaczaniu wartosci kolejnych stanów.
\subsection{wskjak}
Funkcja ta oblicza wartosć wskaźnika jakosci dla zadanych argumentów.

\section{Podsumowanie}
Dla zadanych danych punkt optymalny odnaleziony przez funkcję fmincon to:
\begin{verbatim}
x1=-5.5489;
x2=-0.1905; 
Przy sterowaniu 
u=3.9121;
\end{verbatim}
W przypadku, gdy zakłócenia są zerowe, po uruchomieniu symulacji sterowanie pozostaje w punkcie początkowym - optymalnym, dzięki czemu sprawdzilismy poprawne działanie strategii decyzyjnej.
\subsection{Otrzymane wyniki sprawdzające poprawnosć działania:}
Po uruchomieniu symulacji wyniki prezentowały następujące stany i sterowania:
\begin{verbatim}
TODO: DODAĆ TO CO WYPL?UJE MATLAB
\end{verbatim}
\subsection{Wykres zbieżnosci stanów:}
TODO: DODAĆ OBRAZEK
\end{document}