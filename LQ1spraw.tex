\documentclass[11pt,a4paper]{article}
\usepackage[T1]{fontenc}
\usepackage[polish]{babel}
\usepackage[utf8]{inputenc}
\usepackage{lmodern}
\usepackage{geometry}
\usepackage{enumitem}
\selectlanguage{polish}
\usepackage{graphicx}
\DeclareGraphicsExtensions{.pdf,.png,.jpg}
\usepackage{multicol}


\newgeometry{tmargin=2cm, bmargin=2cm, lmargin=2cm, rmargin=2cm}
\title{Laboratorium SRD\\LQ1}
\date{10 maja 2014}
\author{Katarzyna Olszewska, Łukasz Korpal}
\renewcommand*\thesection{\arabic{section}}
\setcounter{secnumdepth}{5}
\setcounter{tocdepth}{5}
\setlength{\parindent}{0pt}
\begin{document}
\maketitle
\newpage

\section*{Zadanie 8-1}
Sterowany obiekt ma 2 zmienne stanu opisane następującymi równaniami

\begin{displaymath}
x[1]_{k+1} = 1.8x[1]_k + w[1]_k + 0.01 \cdot u[1]_k \\
x[2]_{k+1} = 0.2x[2]_k + 0.3x[1]_k (1 - u[1]_k ) - 5
\end{displaymath}
gdzie
\begin{displaymath}
	x[i]_k - i-ta współrzędna wektora stanu w chwili k 
\end{displaymath}
\begin{displaymath}
u[i]_k - i-ta współrzędna wektora sterowania w chwili k
\end{displaymath}
\begin{displaymath}
w[i]_k - i-ta współrzędna wektora zakłócenia w chwili k
\end{displaymath}
Zakłócenia w kolejnych chwilach są niezależne i mają rozkład stacjowany o wartosci oczekiwanej
\begin{displaymath}
E_w 0 0.1
\end{displaymath}

Celem sterowania jest osiągnięcie jak najmniejszej wartosci oczekiwanej
\begin{displaymath}
J = \sum_{k=0}^{\infty}{F_k(x_{k+1},u_k)}
\end{displaymath}

Gdzie:
\begin{displaymath}
F(x_{k+1}, u_k) = 0.1 (g(u[1]_k))^2 (x[1]_{k+1})^2 + x[2]_{k+1}^2 + 0.1u[1]_k^2
\end{displaymath}
\begin{displaymath}
g(z)=
\end{displaymath}
\begin{displaymath}
	0.4 dla z<0.4
	\end{displaymath}
	\begin{displaymath}
	z dla 0.4 \geq z \geq 1
	\end{displaymath}
	\begin{displaymath}
	1 dla 1<z
\end{displaymath}

\section{Założenia projektowe}
Wartości wskaźników jakości dla stanów i etapów przetrzymywane są w macierzy Jk,
Wartości sterowań dla stanów i etapów przetrzymywane są w macierzy Uk;
Wartości zakłóceń, wraz z odpowiadającym im rozkładem prawdopodobieństwa, został przedstawiony w macierzy PW.

\subsection{Dyskretyzacja}
Przyjęliśmy następujące dyskretyzacje dla poszczególnych wartości:
Sterowania: 15 wartości, d = 300;
Stanu: 20 wartości, d = 400;
Zakłóceń: 7 wartości, 0.02 - 0.05, d = 0.005;

\section{Zakłócenia losowe}
Zakłócenia losowe potrzebne do symulacji generowane są przez funkcję wrand.m.\\
\begin{verbatim}
function [w] = wrand()
    i = rand;
    if i<0.05
        w = 0.02;
    elseif i<0.15
        w = 0.025;
    elseif i<0.35
        w = 0.03;
    elseif i<0.65
        w = 0.035;
    elseif i<0.85
        w = 0.04;
    elseif i<0.95
        w = 0.045;
    else
        w = 0.05;
    end
end
\end{verbatim}

\section{Funkcja główna}
Wszystkie główne zadania realizuje funkcja pd1.m
\subsection{Funkcja w matlabie}
Z uwagi na długość funkcji, została ona załączona w oddzielnym pliku.

\subsection{Opis}
Funkcja ta zawiera dwie części - modelową i symulacyjną.\\
Pierwsza uruchamia algorytm programowania dynamicznego. Uzupełnia początkowe wartości macierzy i uruchamia 4 zagnieżdżone w sobie pętle, wykonujące obliczenia wskaźników jakości dla wszystkich możliwych kombinacji etapów, stanów, sterowań i zakłóceń.\\
Druga uruchamia symulację - dla zadanej ilości obiegów i losowanych wartości zakłóceń z dyskretyzowanego przedziału.
Następnie oblicza średnią wartość wskaźnika jakości dla badanych przypadków, następnie oblicza odchylenie i wartość oczekiwaną wskaźnika jakości.
Ostatecznie, zgromadzone dane wyświetla na wykresach.\\


\section{Podsumowanie}
Otrzymane wyniki przypominają wyniki z doświadczeń z SP - z niewielkimi rozbieżnościami, wynikającymi głównie z przyjętej dyskretyzacji.
Przy współczynnikach A i B porównywalnych (A<B*$10^6$)Student powinien raczej przetrzymać pieniądze (satysfakcja rośnie kwadratowo) i wydać je na końcu rozpatrywanego horyzontu, podnosząc jakość. Dopiero po znacznym zwiększeniu A w stosunku do B obserwujemy, że sterowanie ulega zróżnicowaniu.
\subsection{Otrzymane wyniki:}
\begin{verbatim}
Średnia jakość:
  2.3461e+010

Odchylenie
  420.8904

Wartosc oczekiwana:
  4.5040e+009
\end{verbatim}
\end{document}